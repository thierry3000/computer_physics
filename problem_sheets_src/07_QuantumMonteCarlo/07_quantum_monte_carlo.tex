\documentclass[german,10pt,a4paper]{newexam}
%\usepackage[T1]{fontenc}
\usepackage[utf8]{inputenc}
\usepackage{amsmath}
\usepackage{amssymb}
\usepackage{epsfig}
\usepackage{ngerman}
\usepackage{dsfont}
\usepackage{bm}
\setlength\parskip{\medskipamount}
\setlength\parindent{0pt}

\makeatletter
\usepackage[ngerman]{babel}
\makeatother
\thispagestyle{headandfoot}
\extraheadheight[1.5cm]{0cm}

\firstpageheadrule
\firstpageheader{{\Large \sf Computerphysik}\\{\large \sf SS 17}}
  {\LARGE  7. \"Ubung}
  {{\Large \sf Theoretische Physik\\Universit\"at des Saarlandes}\\ {\large \sf Prof.~Dr.~{\sc Heiko~Rieger} und\\ \large \sf Dr.~{\sc Adam~Wysocki}}}
\footrule
\extrafootheight{0cm}
\lfoot{}
\lfoot{{\bfseries \textsf{Info:}}
{\tt  http://www.uni-saarland.de/fak7/rieger/homepage/teaching.html}\\}
\cfoot{}
\rfoot{\thepage/\numpages}
%\bracketedpoints 
\boxedpoints
%\pointsinrightmargin
\pointsinmargin
\pointpoints{Punkt}{Punkte}
%%%%%%%%%%%%%% define newcommands %%%%%%%%%%%%%%%%%%%%%%%%%%%%%%%%%%%
\newcounter{mycounter}
\setcounter{mycounter}{0}
\newcommand{\myTitle}[1]
   {\addtocounter{mycounter}{1}\qformat{\large \textbf{\themycounter.} {\small \sf [\textsl{\thepoints}]} {\sf \normalsize \textbf{#1}}\hfill}}
\newcommand{\Class}[1]
   {\addtocounter{mycounter}{1}\qformat{\large \textbf{\themycounter.} {\small \sf [\textsl{Pr\"asenzaufgabe}]} {\sf \normalsize \textbf{#1}}\hfill}}
\newcommand{\Remark}[1]{{\large \textsf{Bemerkung:} }{\small \textsf{#1}}}
\newcommand{\Hint}[1]{{\large \textsf{Hinweis:} }{\small \textsf{#1}}}
\newcommand{\Bold}[1]{{\bfseries {#1}}}
\newcommand{\be}{\begin{equation}}
\newcommand{\ee}{\end{equation}}
\newcommand{\ket}[1]{\left| {#1}\right\rangle}
\newcommand{\bra}[1]{\left\langle {#1} \right|}
\newcommand{\skp}[2]{\langle {#1} | {#2}\rangle}
\newcommand{\E}[1]{\left\langle {\hat{#1}}\right\rangle}
\newcommand{\e}[1]{\langle {#1}\rangle}
\newcommand{\trip}[3]{\langle {#1}|{#2}|{#3}\rangle}
\newcommand{\Sz}{\hat S_z}
\newcommand{\Szi}[1]{\hat S_{z #1}}
\newcommand{\SO}{\hat{\vec{S}}}
\newcommand{\zus}[1]{\left| #1 \right>}
%\renewcommand{\theequation}{10.\arabic{equation}}
\newcommand{\beqs}{\begin{eqnarray}}
\newcommand{\eeqs}{\end{eqnarray}}
%%%%%%%%%%%%%%%%%%%%%%%%%%%%%%%%%%%%%%%%%%%%%%%%%%%%%%%%%%%%%%%%%%%%%

\begin{document}
\vspace*{-0.5cm}\hrule
\begin{center} 
{\bfseries \sf(Abgabe: bis zum \textit{5. Juli 2017, 16:00 Uhr}.
Quellcode, Filme und Bilder bitte in ``/home/comphys/comphys\_ss17\_Abgabe/'' im Cip Pool ablegen.
Der schriftliche Teil kann entweder als Pdf beigelegt oder im Postfach von Prof.~Rieger abgegeben werden.
\\
Unter ``/home/comphys/comphys\_ss17/exercises\_supplemental/`` finden sie die jeweilig Dateien, die 
f\"ur die Bearbeitung hilfreich sind.
}\\[.3cm]
\end{center}
\begin{center} 

\Large Quanten-Monte-Carlo Simulation eines eindimensionalen Bose-Hubbard-Modells\\[.3cm]
\normalsize

\end{center}


Die Datei bosehubbard.cpp simuliert ein 
eindimensionales Bose-Hubbard-Modell mit periodischen Randbedingungen im Grundzustand.
In der Präsenzübung wird die Implementierung des Algorithmus in der Datei 
bosehubbard.cpp erläutert. Die wichtigsten Klassen sind:
	\begin{itemize}
 		\item checkerboard: Verwaltung der checkerboard decomposition, 
 		Ausführen von QMC-Updates, Ausgabe der aktuellen Konfiguration
		\item correlationtime: Betrachten der Korrelationen zwischen den einzelnen Samples
		%\item currentcorrelation: Ausgabe der Pseudo-Strom-Strom-Korrelationsfunktion zur Detektion von Superfluidem und Mott-Isolator Zustand
		\item matrixelements: Berechnung der Gewichte der Plaquetten
		\item matrixelements2: Berechnung von Beiträgen der kinetischen Energie der einzelnen Plaquetten
	\end{itemize}

Der Hamiltonoperator des betrachteten Systems lautet: 
\begin{equation}
  H = -t \sum_{<ij>}(a_j^+ a_i + a_i^+ a_j) - \mu \sum_i n_i + V_0 \sum_i n_i(n_i-1)
\end{equation}
Die erste Summe läuft dabei nur über nächste Nachbargitterplätze. 
Wir betrachten im folgenden Systeme mit der inversen Temperatur $\beta = 2$, 
dem Hoppingparameter $t = 1$ und dem Onsitepotential $V_0 = 20$. 
Änderungen der Parameter werden explizit angegeben. 
Die Diskretisierung der Imaginärzeit wählen wir als $\delta \tau = 0.0625$, was 32 Dikretisierungspunkten entspricht.
Es sollen 16 Gitterplätze betrachtet werden.

\begin{questions}


\vspace{0.3cm}

\myTitle{Korrelationszeit}
\question[1]
	Lassen Sie sich die Autokorrelationsfunktion für obiges 
	System mit 16 Bosonen ausgeben. 
	Bestimmen Sie die Korrelationszeit zwischen unabhängigen Samples.

\myTitle{kinetische Energie pro Gitterplatz}
\question[2,5]
	Wir definieren eine Bosonendichte $\rho$ als Quotient aus 
	Bosonenzahl $N_b$ und der Zahl der Gitterplätze $N$.
\begin{parts}
 \part Die Klasse kinetic berechnet die mittlere kinetische Energie 
 \textbf{eines Samples}. Benutzen Sie die Klasse kinetic um die mittlere kinetische 
 Energie pro Gitterplatz $E_K$ für Bosonendichten $\rho$ zwischen $0$ und $3.2$ zu berechnen 
 und plotten Sie $- E_K$ normiert auf das Onsitepotential $V_0$ in Abhängigkeit von $\rho$. \\
\Hint{Mitteln Sie über eine hinreichend große Anzahl von Monte Carlo Samples.}
 \part An welchen Stellen finden Sie Minima? Liegt an den Minima ein superfluider oder ein Mott-Isolator Zustand vor? Wie sieht es an Stellen dazwischen aus?
\end{parts}
\pagebreak_
\myTitle{chemisches Potential}
\question[3]
	Für $\beta = 2$ können thermische Anregungen vernachlässigt werden 
	und es ist ausreichend das System im Grundzustand zu betrachten. 
	Das chemische Potential $\mu$ ergibt sich aus der Ableitung der Grundzustandsenergie 
	nach der Bosonenzahl. Hier erhalten wir also für das chemische Potential 
	$\mu = E_{N_b+1} - E_{N_b}$, wobei $E_{N_b}$  die Grundzustandsenergie eines Systems mit $N_b$ Bosonen ist.
\begin{parts}
	\part Plotten Sie die Grundzustandsenergie $E_{N_b}$ 
	des Systems in Abhängigkeit von der Bosonendichte $\rho$. 
	Was fällt bei dem Plot auf ?
	\part Berechnen Sie nun das chemische Potential $\mu$ 
	und plotten Sie dieses dann wieder in Abhängigkeit von der 
	Bosonenzahldichte $\rho$. Wie äußert sich das Verhalten aus Teil a) in diesem Plot ?
\end{parts}

\myTitle{Phasendiagramm}
\question[3,5]

In dieser Aufgabe sollen Sie nun das Phasendiagramm für den Phasenübergang 
vom Mott-Isolator zum superfluiden Zustand bestimmen. 
Im Phasendiagramm plotten Sie das chemische Potential $\mu$, 
welches mit dem Onsitepotential $V_0$ normiert wird, gegen den Hopping-Parameter $t$, 
welcher ebenfalls mit $V_0$ normiert wird. 
Überlegen Sie sich, wie Sie mit dem Code der vorangegangenen Aufgaben 
das Phasendiagramm erstellen können. Ihr Ergebnis sollte Abbildung \ref{fig:phase} entsprechen.
\end{questions}

\begin{figure}[hb]
 \centering
 \includegraphics[width = 0.75 \textwidth]{phasendiagramm-eps-converted-to.pdf}
 \caption{Phasendiagramm Mott-Isolator - Superfluider Zustand}
 \label{fig:phase}
\end{figure}



\end{document}



