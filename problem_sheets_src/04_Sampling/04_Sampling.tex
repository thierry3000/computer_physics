\documentclass[english,10pt]{newexam}
%\usepackage[T1]{fontenc}
\usepackage[utf8]{inputenc}
\usepackage{amsmath}
\usepackage{amssymb}
\usepackage{epsfig}
\usepackage{ngerman}
\usepackage{dsfont}
\usepackage{bm}
\usepackage{bbm}
\usepackage{color}
\setlength\parskip{\medskipamount}
\setlength\parindent{0pt}

\makeatletter
\usepackage[ngerman]{babel}
\makeatother
\thispagestyle{headandfoot}
\extraheadheight[1.5cm]{0cm}

\firstpageheadrule
\firstpageheader{{\Large \sf Computerphysik}\\{\large \sf  SS 2017}}
  {\LARGE  4. \"Ubung - Sampling}
  {{\Large \sf Theoretische Physik\\Universit\"at des Saarlandes}\\ {\large \sf Prof.~Dr.~{\sc Heiko~Rieger} und\\ \large \sf Dr.~{\sc Adam~Wysocki}}}
\footrule
\extrafootheight{-2cm}
\lfoot{}
\lfoot{{\bfseries \textsf{Info:}}
{\tt  http://www.uni-saarland.de/fak7/rieger/homepage/teaching.html}\\}
\cfoot{}
\rfoot{\thepage/\numpages}
%\bracketedpoints 
\boxedpoints
%\pointsinrightmargin
\pointsinmargin
\pointpoints{Punkt}{Punkte}
%%%%%%%%%%%%%% define newcommands %%%%%%%%%%%%%%%%%%%%%%%%%%%%%%%%%%%
\newcounter{mycounter}
\setcounter{mycounter}{0}
\newcommand{\myTitle}[1]
   {\addtocounter{mycounter}{1}\qformat{\large \textbf{\themycounter.} {\small \sf [\textsl{\thepoints}]} {\sf \normalsize \textbf{#1}}\hfill}}
\newcommand{\Class}[1]
   {\addtocounter{mycounter}{1}\qformat{\large \textbf{\themycounter.} {\small \sf [\textsl{PrÀsenzaufgabe}]} {\sf \normalsize \textbf{#1}}\hfill}}
\newcommand{\Remark}[1]{{\large \textsf{Bemerkung:} }{\small \textsf{#1}}}
\newcommand{\Hint}[1]{{\large \textsf{Hinweis:} }{\small \textsf{#1}}}
\newcommand{\Bold}[1]{{\bfseries {#1}}}
\newcommand{\be}{\begin{equation}}
\newcommand{\ee}{\end{equation}}
\newcommand{\ket}[1]{\left| {#1}\right\rangle}
\newcommand{\bra}[1]{\left\langle {#1} \right|}
\newcommand{\skp}[2]{\langle {#1} | {#2}\rangle}
\newcommand{\E}[1]{\left\langle {\hat{#1}}\right\rangle}
\newcommand{\e}[1]{\langle {#1}\rangle}
\newcommand{\trip}[3]{\langle {#1}|{#2}|{#3}\rangle}
\newcommand{\Sz}{\hat S_z}
\newcommand{\Szi}[1]{\hat S_{z #1}}
\newcommand{\SO}{\hat{\vec{S}}}
\newcommand{\zus}[1]{\left| #1 \right>}
%\renewcommand{\theequation}{10.\arabic{equation}}
\newcommand{\beqs}{\begin{eqnarray}}
\newcommand{\eeqs}{\end{eqnarray}}
%%%%%%%%%%%%%%%%%%%%%%%%%%%%%%%%%%%%%%%%%%%%%%%%%%%%%%%%%%%%%%%%%%%%%

\begin{document}
\vspace*{-0.5cm}\hrule
\begin{center} 
{\bfseries \sf(Abgabe: bis zum \textit{{\color{red} 31. Mai 2017, 16:00 Uhr}})\\
Quellcode, Filme und Bilder bitte in ``/home/comphys/comphys\_ss17\_Abgabe/'' im Cip Pool ablegen.
Der schriftliche Teil kann entweder als Pdf beigelegt oder im Postfach von Prof.~Rieger abgegeben werden.
\\
Unter ``/home/comphys/comphys\_ss17/exercises\_supplemental/`` finden sie die jeweilig Dateien, die 
f\"ur die Bearbeitung hilfreich sind.}\\[.3cm]
\end{center}

\begin{questions}


\vspace{0.3cm}

\myTitle{Tower Sampling}
\question[2]
Schreiben Sie ein Testprogramm, welches die Funktion ''int TowerSampling()'' beinhaltet. Das Tower Sampling soll Ihnen dabei helfen, 
eine m\"ogliche Freizeitbesch\"aftigung f\"ur Ihren Samstagabend zu finden.
Betrachten Sie dabei folgende Aktivit\"aten mit jeweils der Wahrscheinlichkeit $P$: Lernen ($P=0.02$), Freunde treffen ($P=0.35$), Kino ($P=0.1$),
Buch lesen ($P=0.2$), Sport treiben ($P=0.05$), DVD anschauen ($P=0.15$), Kochen ($P=0.13$). W\"urfeln Sie a) 1.000 b) 1.000.000 mal eine Freizeitaktivit\"at aus und 
\"uberzeugen Sie sich von der korrekten Arbeitsweise Ihres Testprogramms. 

\myTitle{Reaktionskinetik mit Gillespie}
\question[4]
 Gegeben sei nachfolgendes Reaktionsschema für die fünf Spezies $A$, $B$, $C$ , $D$ und $E$: 
\begin{itemize}
 \item $ A + B \underset{c_{2}}{\overset{c_{1}}{\rightleftarrows}} C$
 \item $ C + D \underset{c_{4}}{\overset{c_{3}}{\rightleftarrows}} E$
\end{itemize}
\begin{parts}
\part Implementieren Sie Gillespies Algorithmus für beliebige Raten $k_1$, $k_2$, $k_3$, $k_4$ und Anfangsteilchenzahlen $N_A$, $N_B$, $N_C$, $N_D$, $N_E$.     
\part Untersuchen Sie für $c_1=1$, $c_2=1.5$, $c_3=2$, $c_4=2.5$ die beiden Fälle  
\begin{itemize}
 \item[i)] $N_E=10^6$, $N_A=N_B=N_C=N_D=0$
 \item[ii)] $N_E=10^2$, $N_A=N_B=N_C=N_D=0$
 \end{itemize}
Gehen Sie dabei jeweils folgendermaßen vor: Stellen Sie $n_i(t)=N_i(t)/{N_E}$ für $i\in{A,B,C,D,E}$ gemittelt über 1000 Samples graphisch dar. Was fällt Ihnen auf? \\
Anschließend sollen nun relative ``Sample-to-Sample-Fluktuationen`` untersucht werden. Stellen Sie dazu $\Delta(t)=n^{S_1}_i(t)-n^{S_2}_i(t)$ gemittelt über 1000 Samplepaare $(S_1,S_2)$ für alle $i$ in beiden Fällen graphisch dar. Erklären Sie die Graphen.
\end{parts}


\myTitle{''Next-Event-Sampling`` nach Gillespie für zeitabhängige Eventdichten}
\question[4]
In der Vorlesung haben Sie Gillespies Algorithmus für konstante Übergangsraten $k_i$ kennengelernt, d.h. die Wahrscheinlichkeitsdichte $\rho_i(t)$ für ein isoliertes Event $i$ ist dort gegeben durch eine Exponentialverteilung mit Parameter $k_i$. Die Wahrscheinlichkeitsdichte $\rho_{next}(t)$ für das nächste Event ist gegeben durch eine Exponentialverteilung mit Parameter $k=\sum_i k_i$. Die Korrektheit des Gillespie Algorithmus knüpft nicht an diese spezielle Form der Dichten. Dies soll Ihnen mithilfe dieser Aufgabe verdeutlicht werden. \\
Gegeben seien dazu N beliebige Wahrscheinlichkeitsdichten $\rho_i: \mathbb{R}^+_0 \rightarrow \mathbb{R}^+_0$ ($i=1,...,N$) mit zugehörigen Verteilungsfunktionen $F_i(t)$.
\begin{parts}
\part Zeigen Sie, dass 
\begin{equation}
\rho_{next}(t)=\sum_{i}\rho_i(t)\prod_{j\neq i}(1-F_i(t))
\end{equation}
die Wahrscheinlichkeitsdichte (in der Zeit) für das erste Event zum Zeitpunkt $t$ ist und damit  
\begin{equation}
\varrho(i,t)=\rho_i(t)\prod_{j\neq i}(1-F_j(t))
\end{equation}
die Wahrscheinlichkeitsdichte (in Eventart und Zeit) ist, dass zum Zeitpunkt $t$ das erste Event stattfindet und von Typ $i$ ist. \newpage
\part Begründen Sie, dass nachstehender Gillespie Algorithmus korrekt Tupel (i,t) gemäß $\varrho(i,t)$ erzeugt.
\begin{itemize}
 \item for $(j=1..N)$ do: $\tau_j=$ random number according to $\rho_j$ 
 \item $\tau=\text{min}\{\tau_j\}_{j=1..N}$ und $i$ zugehöriger Index 
 \item return $(i,\tau)$
 \end{itemize}
\part Implementieren Sie die Routine aus Aufgabenteil b) für $N=2$ mit einer 
exponentialverteilten Dichte $\rho_1(t)=2e^{-\frac{1}{2}t}$ 
und der linear verteilten Dichte $\rho_2(t)=\frac{2}{a^2}t$ (auf (0,a), sonst identisch Null)
mit a=2 (Aufgabe 1 Blatt 3). Erzeugen Sie ein Dichtehistogramm mit $10^8$ Samples für $\rho_{next}(t)$
und zeichnen Sie die analytische Kurve darüber.
\end{parts}



\end{questions}


Bei Fragen E-Mail an: thierry@lusi.uni-sb.de 
\end{document}




