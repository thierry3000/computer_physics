\documentclass[german,10pt,a4paper]{newexam}
%\usepackage[T1]{fontenc}
\usepackage[utf8]{inputenc}
\usepackage{amsmath}
\usepackage{amssymb}
\usepackage{epsfig}
\usepackage{ngerman}
\usepackage{dsfont}
\usepackage{bm}
\usepackage{braket}
\usepackage{bbm}
\usepackage{units}
\usepackage{braket}

\newcommand{\kreis}[1]{\unitlength1ex\begin{picture}(2.5,2.5)%
\put(0.75,0.75){\circle{2.5}}\put(0.75,0.75){\makebox(0,0){#1}}\end{picture}}

\setlength\parskip{\medskipamount}
\setlength\parindent{0pt}

\makeatletter
\usepackage[ngerman]{babel}
\makeatother
\thispagestyle{headandfoot}
\extraheadheight[1.5cm]{0cm}

\firstpageheadrule
\firstpageheader{{\Large \sf Computerphysik}\\{\Large \sf WS 2014/15}}
  {{\LARGE  8. {\"U}bung}\\{}}
  {{\Large \sf Theoretische Physik\\Universit{\"a}t des Saarlandes}\\ {\Large \sf Prof.~Dr.~{\sc Heiko~Rieger}}}
\vspace*{0,2cm}
\footrule
\extrafootheight{0cm}
\lfoot{}
\cfoot{}
\rfoot{\thepage/\numpages}
%\bracketedpoints 
\boxedpoints
%\pointsinrightmargin
\pointsinmargin
\pointpoints{Punkt}{Punkte}
%%%%%%%%%%%%%% define newcommands %%%%%%%%%%%%%%%%%%%%%%%%%%%%%%%%%%%
\newcounter{mycounter}
\setcounter{mycounter}{0}
\newcommand{\myTitle}[1]
   {\addtocounter{mycounter}{1}\qformat{\large \textbf{\themycounter.} {\small \sf [\textsl{\thepoints}]} {\sf \normalsize \textbf{#1}}\hfill}}
\newcommand{\myTitleBonus}[1]
   {\qformat{\large \textbf{*} {\small \sf [\textsl{\thepoints}]} {\sf \normalsize \textbf{#1}}\hfill}}
\newcommand{\Class}[1]
   {\addtocounter{mycounter}{1}\qformat{\large \textbf{\themycounter.} {\small \sf [\textsl{Pr{\"a}senzaufgabe}]} {\sf \normalsize \textbf{#1}}\hfill}}
\newcommand{\Remark}[1]{{\large \textsf{Bemerkung:} }{\small \textsf{#1}}}
\newcommand{\Hint}[1]{{\large \textsf{Hinweis:} }{\small \textsf{#1}}}
\newcommand{\Bold}[1]{{\bfseries {#1}}}

%%%%%%%%%%%%%%%%%%%%%%%%%%%%%%%%%%%%%%%%%%%%%%%%%%%%%%%%%%%%%%%%%%%%%

\begin{document}

\vspace*{-1cm}\hrule
\begin{center} 
{\bfseries \sf Ihre L{\"o}sung ist bis zum 11.02.2015 um 12:00 Uhr einzureichen. Legen Sie dazu Ihren Quellcode sowie die anzufertigenden Graphen im Ordner
/home/comphys/comphys\_ws1415\_Abgabe/ im CIP-Pool der Physik ab.}\\[0.3cm]
\end{center}

Es werde eine geschlossene Spinkette mit einer geraden Anzahl $L$ an Spins betrachtet, welche durch den Heisenberg-Hamiltonoperator
\begin{equation*}
H=J\sum_{i=1}^{L}\vec{S}_{i}\cdot\vec{S}_{i+1}=J\sum_{i=1}^{L}\left[\dfrac{1}{2}\left(S_{i}^{+}S_{i+1}^{-}+S_{i}^{-}S_{i+1}^{+}\right)+S_{i}^{z}S_{i+1}^{z}\right]
\end{equation*}
beschrieben wird. Aufgrund der periodischen Randbedingungen gilt $\vec{S}_{i+L}=\vec{S}_{i}$. F{\"u}r $J>0$ liegt eine antiferromagnetische Wechselwirkung zwischen den Spins vor.\\
Als Basisvektoren k{\"o}nnen die Vektoren
\begin{center}
$\ket{K}=|s_1,s_2,\dots,s_L\rangle$
\end{center}
gew{\"a}hlt werden, wobei $s_{i}\in\left\{-S,-S+1,\dots,S-1,S\right\}$. $S$ ist hierbei die Spinquantenzahl. Die Dimension des Hilbertraumes ergibt sich somit zu $D=\left(2S+1\right)^{L}$.\\
Zur Numerierung der Zust{\"a}nde der Kette verschiebt man die Werte der $s_{i}$ um $S$, sodass $s_{i}\in\left\{0,1,\dots,2S\right\}$. Jedem Zustand $\ket{K}$ kann dann die Nummer
\begin{equation*}
K=\sum_{i=1}^{L}s_{i}M_{S}^{i-1}
\end{equation*}
zugeordnet werden, wobei $M_{S}=2S+1$.\\
Die Energiel{\"u}cke zwischen dem Grundzustand und dem erstem angeregtem Zustand verschwindet f{\"u}r unendlich lange Ketten und verringert sich infolgedessen mit wachsender Systemgr{\"o}{\ss}e.
Diese Verringerung soll im Folgenden untersucht werden.

\begin{questions}

\myTitle{Lanczos-Algorithmus}
\question[5]
Implementieren Sie in C++ den Lanczos-Algorithmus zur Berechnung der Grundzustandsenergie sowie der Energie des ersten angeregten Zustandes f{\"u}r eine durch den Heisenberg-Hamiltonoperator
beschriebene geschlossenen Spinkette der L{\"a}nge $L$ mit Spinquantenzahl $S$. Hierbei sei $J=1.0$.

\myTitle{Spin-$1/2$-Kette}
\question[2.5]
Berechnen Sie mithilfe des in der vorherigen Aufgabe implementierten Lanczos-Algorithmus die Grundzustandsenergie und die Energie des ersten angeregten Zustandes einer geschlossenen antiferromagnetischen
Spin-$1/2$-Kette der L{\"a}nge $L\in\left\{8,10,12,14,16,18,20,22\right\}$. Tragen Sie sowohl $E_0$, $E_1$ als auch $\Delta E=E_{1}-E_{0}$ gegen $L$ auf.\\
Zum Testen Ihres Programmes: Eine geschlossene Kette mit $12$ Gitterpl{\"a}tzen besitzt die Grundzustandsenergie $E_{0}=-5.38739$.

\myTitle{Spin-$1$-Kette}
\question[2.5]
Untersuchen Sie nun die Grundzustandsenergie, die Energie des ersten angeregten Zustandes sowie die sich ergebende Energiel{\"u}cke f{\"u}r eine geschlossene antiferromagnetische Spin-$1$-Kette.
Betrachten Sie hierzu Ketten der L{\"a}nge $L\in\left\{6,8,10,12,14,16\right\}$. Welchen Unterschied zur Spin-$1/2$-Kette stellen Sie fest?

\end{questions}

\newpage

\textbf{Hilfe:}\\
Der Algorithmus zur Berechnung der Tridiagonalmatrix lautet folgenderma{\ss}en:\\\\
\textbf{Choose initial} $|r\rangle$\\
$b_0=|||r\rangle||$\\
$|q_0\rangle=0$\\
$|q_1\rangle=|r\rangle/b_0$\\
\textbf{FOR} j=1 \textbf{TO} m \textbf{DO}\\
\hspace*{1em}$|u\rangle=\textbf{H}|q_j\rangle$\\
\hspace*{1em}$|r\rangle=|u\rangle-b_{j-1}|q_{j-1}\rangle$\\
\hspace*{1em}$a_j=\langle q_j|r\rangle$\\
\hspace*{1em}$|r\rangle=|r\rangle-a_{j}|q_j\rangle$\\
\hspace*{1em}\textbf{Reorthogonalize( $|r\rangle$ )}\\
\hspace*{1em}$b_{j}=|||r\rangle||$\\
\hspace*{1em}$|q_{j+1}\rangle=|r\rangle/b_j$\\
\textbf{END FOR}\\

Wichtig f{\"u}r diesen Algorithmus ist die Matrix-Vektor-Multiplikation $H|\psi\rangle$, die folgenderma{\ss}en implementiert werden kann:\\\\
\textbf{FOR} each site $i$ \textbf{DO}\\
\hspace*{1em}\textbf{FOR} $N1=0$ \textbf{TO} $M_{S}^{i-1}-1$ \textbf{DO}\\
\hspace*{2em}\textbf{FOR} $N2=0$ \textbf{TO} $M_{S}^{L-i-1}-1$ \textbf{DO}\\
\hspace*{3em}\textbf{FOR} $S1=0$ \textbf{TO} $M_S-1$ \textbf{DO}\\
\hspace*{4em}\textbf{FOR} $S2=0$ \textbf{TO} $M_S-1$ \textbf{DO}\\
\hspace*{5em}\textbf{FOR} $S1P=0$ \textbf{TO} $M_S-1$ \textbf{DO}\\
\hspace*{6em}\textbf{FOR} $S2P=0$ \textbf{TO} $M_S-1$ \textbf{DO}\\
\hspace*{7em}$K1=N1+N2\cdot M_{S}^{i+1}+S1\cdot M_{S}^{i-1}+S2\cdot M_{S}^{i}$\\
\hspace*{7em}$K2=N1+N2\cdot M_{S}^{i+1}+S1P\cdot M_{S}^{i-1}+S2P\cdot M_{S}^{i}$\\
\hspace*{7em}$HPSI(K2)=HPSI(K2)+H_{MAT}(S1P,S2P,S1,S2)\cdot PSI(K1)$\\
\hspace*{6em}\textbf{END FOR}\\
\hspace*{5em}\textbf{END FOR}\\
\hspace*{4em}\textbf{END FOR}\\
\hspace*{3em}\textbf{END FOR}\\
\hspace*{2em}\textbf{END FOR}\\
\hspace*{1em}\textbf{END FOR}\\
\textbf{END FOR}\\\\
Die Matrix $H_{MAT}$ bezeichnet den Teil des Hamiltonoperators, der den Spin am Gitterplatz $i$ mit dem Nachbarspin $i+1$ koppelt.
Beachten Sie, dass f{\"u}r $i$ nahe des Randes kleine Modifikationen vonn{\"o}ten sind.

\end{document}