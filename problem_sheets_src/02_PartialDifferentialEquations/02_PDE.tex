\documentclass{exam}
\usepackage[utf8]{inputenc}
\usepackage{amsmath}
\usepackage{amsfonts}
\usepackage{amssymb}

\extraheadheight[1.5cm]{0cm}

\firstpageheadrule
\firstpageheader{{\Large \sf Computerphysik}\\{\large \sf  WS 2018/2019}}
  {\LARGE  2. \"Ubung - PDE}
  {{\Large \sf Theoretische Physik\\Universit\"at des Saarlandes}\\ {\large \sf Prof.~Dr.~{\sc Heiko~Rieger}}}
  
\footrule
\extrafootheight{-2cm}
\lfoot{}
\lfoot{{\bfseries \textsf{Info:}}
{\tt  http://www.uni-saarland.de/fak7/rieger/homepage/teaching.html}\\}
\cfoot{}
\rfoot{\thepage/\numpages}
\boxedpoints
\pointsinmargin
\pointpoints{Punkt}{Punkte}

% %%%%%%%%%%%%%% define newcommands %%%%%%%%%%%%%%%%%%%%%%%%%%%%%%%%%%%
\newcounter{mycounter}
\setcounter{mycounter}{0}
\newcommand{\myTitle}[1]
   {\addtocounter{mycounter}{1}\qformat{\large \textbf{\themycounter.} {\small \sf [\textsl{\thepoints}]} {\sf \normalsize \textbf{#1}}\hfill}}
\newcommand{\Class}[1]
   {\addtocounter{mycounter}{1}\qformat{\large \textbf{\themycounter.} {\small \sf [\textsl{PrÀsenzaufgabe}]} {\sf \normalsize \textbf{#1}}\hfill}}
\newcommand{\Remark}[1]{{\large \textsf{Bemerkung:} }{\small \textsf{#1}}}
\newcommand{\Hint}[1]{{\large \textsf{Hinweis:} }{\small \textsf{#1}}}
\newcommand{\Bold}[1]{{\bfseries {#1}}}
\newcommand{\be}{\begin{equation}}
\newcommand{\ee}{\end{equation}}
\newcommand{\ket}[1]{\left| {#1}\right\rangle}
\newcommand{\bra}[1]{\left\langle {#1} \right|}
\newcommand{\skp}[2]{\langle {#1} | {#2}\rangle}
\newcommand{\E}[1]{\left\langle {\hat{#1}}\right\rangle}
\newcommand{\e}[1]{\langle {#1}\rangle}
\newcommand{\trip}[3]{\langle {#1}|{#2}|{#3}\rangle}
\newcommand{\Sz}{\hat S_z}
\newcommand{\Szi}[1]{\hat S_{z #1}}
\newcommand{\SO}{\hat{\vec{S}}}
\newcommand{\zus}[1]{\left| #1 \right>}
%\renewcommand{\theequation}{10.\arabic{equation}}
\newcommand{\beqs}{\begin{eqnarray}}
\newcommand{\eeqs}{\end{eqnarray}}
% %%%%%%%%%%%%%%%%%%%%%%%%%%%%%%%%%%%%%%%%%%%%%%%%%%%%%%%%%%%%%%%%%%%%%

\begin{document}
\vspace*{-0.5cm}\hrule
\begin{center} 
{\bfseries \sf(Abgabe und Besprechung am 15.11.2018 um 14:00 im CIP-Pool)}\\[.3cm]
\end{center}

Ziel dieser Übung ist es eine numerische Lösung einer Variante des Räuber-Beute Modells zu berechnen.
Das betrachtete Modell beinhaltet nun eine Ortsabhängigkeit der Räuber-und
Beutekonzentration, genannt  $v(x, y, t)$ bzw. $u(x, y, t)$ in zwei Dimensionen, wobei $x$ und $y$ die beiden Ortskoordinaten
bzeichnen. Die zeitliche Entwicklung wird durch das folgende
Differentialgleichungssystem beschrieben:
\begin{eqnarray}
\label{eqn:rb}
\frac{\partial u}{\partial t} &=& \delta_1 \Delta u + r u \left(1 - \frac{u}{w} \right) - p v \Psi(k u) \\
\frac{\partial v}{\partial t} &=& \delta_2 \Delta v + q v \Psi(k u) - s v \nonumber \\
\Psi &:=& \eta \mapsto \tfrac{\eta}{1+\eta} \nonumber\text{,}
\end{eqnarray}
wobei $\delta_1, \delta_2, r, p, k, q, s$ positive Modellparameter sind und 
$\Delta$ den Laplace Operator bezeichnet $\Delta = 
\frac{\partial^2}{\partial x^2} + \frac{\partial^2}{\partial y^2}$. Wir wollen 
die Gleichung auf einem quadratischen Gebiet $\Omega = [0, L]^2$ der Seitenlänge 
$L$ lösen. Um die Definition des Systems zu vervollständigen wählen wir 
Neumannsche Randbedingungen: $\partial u/\partial {\bf n} = \partial v/\partial 
{\bf n} = 0$, wobei ${\bf n}$ den Normalenvektor am Rand des Simulationsgebietes 
bezeichnet.

Zur numerischen Lösung diskretisieren wir das Simulationsgebiet durch ein regelmäßiges Gitter mit
Gitterkonstante $h$. Das bedeutet, eine Funktion $f(x, y)$ ist durch ihre Werte an den Gitterplätzen
$f(i h, j h) = f_{i,j}$ eindeutig bestimmt, wobei $i,j = 0,\dots,L/h$. Es läßt sich zeigen, daß die
Anwendung des Laplace Operators folgendermaßen approximiert werden kann:
\begin{equation}
\label{eqn:dlaplacian}
(\Delta f)_{i,j} \approx (\Delta_h f)_{i,j} = \frac{1}{h^2} [ -4 f_{i,j} + f_{i-1,j} + f_{i,j-1} + f_{i+1,j} + f_{i,j+1} ]\text{,}
\end{equation}
wobei $\Delta_h$ den diskreten Laplace Operator bezeichnet.
Die Randbedingungen benötigen dabei spezielle Behandlung. Betrachten wir zunächst den linken Rand, und
stellen uns vor es existieren noch Gitterpunkte für $i = -1$. Wegen der Randbedingung, gilt dann an
jedem Platz $(0,j)$: $\partial f/\partial x \approx (f_{1,j} - f_{-1,j})/2h = 0$. Damit läßt sich $f_{-1,j}$
aus $(\Delta_h f)_{0,j}$ eleminieren. Analoges gilt für die anderen Ränder.

Die Zeit diskretisieren wir in konstante Schritte der Länge $\tau$ und schreiben
für eine Funktion der Zeit am $n$-ten Zeitschritt $f(n \tau) = f^n$. Die zeitliche Ableitung diskretisieren
wir mit der Eulermethode. Sei $\partial f/\partial t = F[f]$ eine Differentialgleichung
mit dem Operator $F$. Dann wird $f^{n+1}$ gegeben durch
\begin{equation}
\label{eqn:dtime}
\frac{f^{n+1} - f^{n}}{\tau} = F[f^{n}]
\end{equation}
% Die Bedeutung von explizit ist, daß nur Werte von bereits vergangenen Zeitschritten auf der rechten Seite
% von GL. (\ref{eqn:dtime}) auftauchen. Bei impliziten Verfahren führt man stattdessen formal den zu berechnenden
% Zeitpunkt auf der rechten Seite ein: $F[f^{n+1}]$, was auf die Lösung eines linearen oder nicht-linearen Gleichungssytems hinausläuft.

% Für die meisten sog. expliziten Verfahren existieren Stabilitätskriterien, die die Länge des Zeitschrittes
% $\tau$ bei gegebenen anderen Parametern beschränken. Es lässt sich zeigen, daß für ein Reaktions-Diffusions
% System wie GL. (\ref{eqn:rb}) gelten muss:
% \begin{equation}
% \label{eqn:stab}
% \frac{D d 2 \tau}{h^2} < 1, \text{,}
% \end{equation}
% mit der Diffusionskonstante $D$ und Raumdimension $d$.

Der Zeitschritt für das diskretisierte Räuber-Beute System (\ref{eqn:rb}) sieht also folgendermaßen aus:
\begin{eqnarray}
\label{eqn:drb}
u_{i,j}^{n+1}  &=& u_{i,j}^n + \tau \left[ \delta_1 (\Delta_h u^n)_{i,j} + r u^n_{i,j} \left(1 - \frac{u^n_{i,j}}{w} \right) - p v^n_{i,j} \Psi(k u^n_{i,j}) \right] \\
v_{i,j}^{n+1}  &=& v_{i,j}^n + \tau \left[ \delta_2 (\Delta_h v^n)_{i,j} + q v^n_{i,j} \Psi(k u^n_{i,j}) - s v^n_{i,j} \right] \notag 
\end{eqnarray}

\clearpage

\begin{questions}
\vspace{0.3cm}

% \myTitle{Ein bisschen Theorie}
% \question[2]
% Leiten sie mit Hilfe der Taylorentwicklung die Approximation des Laplace Operators in 1d
% $\Delta f = \frac{1}{h^2}(- 2 f_i + f_{i-1} + f_{i+1})$ her und bestimmen sie die Ordnung des
% Fehlers in $h$.
\myTitle{Ein bischen Theorie}
\question[3]
\begin{parts}
\part Leiten Sie mit Hilfe der Taylorentwicklung die Approximation des Laplace Operators 
in einer Dimension her: 
$\Delta f \approx \frac{1}{h^2}(- 2 f_i + f_{i-1} + f_{i+1})$ und bestimmen sie die Ordnung des
Fehlers in $h$.
\part Geben sie eine plausible Erklärung (oder Beweis wenn Sie wollen) für das Stabilitätskriterium
$\frac{2 D \tau}{h^2}\leq 1$ an. Hinweis: negative Konzentrationswerte sind physikalisch unsinnig.
\end{parts}

\myTitle{Programmierung: Finite Differenzen und Euler Verfahren}
\question[6]
Implementieren Sie die Zeitintegration wie in GL. (\ref{eqn:drb}) angegeben.
Vervollständigen sie dazu den Rumpf der {\it integrate} Funktion.
Verwenden sie die Approximation für den Laplace Operator wie in GL. (\ref{eqn:dlaplacian}). 

\myTitle{Nichlineare Dynamik}
\question[1]
Ziel dieser Aufgabe ist es, Ihr Program mit Parametern zu testen, die eine interessante Dynamik
erzeugen. Probieren Sie $\delta_1 (\text{im Code d1}) = \delta_2(\text{im Code d2}) = r = p = w = q = 1, s = 1/2, k = 5, L = 200, \tau = 1/10, h = 1$.
Erzeugen Sie dann aus den Datendateien eine Bilderserie und daraus wiederum ein Video. Hinweis: Es sollten
Spiralwellen zu sehen sein.
\end{questions}
Infos und aktuelle \"Ubungsbl\"atter finden Sie unter: http://www.uni-saarland.de/fak7/rieger/homepage/teaching.html
Bei Fragen E-Mail an: a.wysocki@lusi.uni-sb.de
\end{document}



